\documentclass[a4paper,12pt,twocolumn]{article}
\usepackage{picinpar,graphicx}
	\usepackage{indentfirst}
		\usepackage{cite}
			\usepackage{multirow}
				\usepackage{CJK}
				
	\setlength{\parindent}{3em}
\setlength{\parskip}{1em}
\pagestyle{plain}
\linespread{1.5}
\begin{document}
\begin{center}
	
	\textbf{\bfseries \LARGE 1 to 1 Internet Platform} 
\end{center}
\begin{center}
	Lina Zhu
\end{center}
\begin{center}
	\today
\end{center}

	\par 	This article tells us that one of the major goals of the education system is to improve student achievement and satisfaction. One of the major goals of the education system is to improve student achievement and satisfaction. In order to be able to tailor the teaching process according to each student's needs and preferences, the teacher must accurately assess the students' different abilities. This will naturally vary in terms of knowledge level, interests, social background and motivation level. Studies have shown that one-on-one teaching is more likely to produce higher student learning performance than group education. However, providing such attention and teaching in traditional classrooms can be difficult.
	\par The Internet has become a central core of the educational environment experienced by learners, which helps to learn anytime, anywhere. Allen and Seaman claim that in 2008 about a quarter of secondary and higher education students in the United States were undergoing full online courses. By 2009, Ambient Insight Research reported that 44\% of post-secondary students in the United States study at least some courses online and expect to increase their penetration rate to 81\% by 2014. As a result, advanced economies have been the main market for self-learning e-learning products. However, due to the substantial increase in the number of suppliers, the developing economy has shown a keen interest in e-learning.
	\par The global e-learning system is beginning to define. In 2011, the global self-learned e-learning market reached a total of 35.6 billion U.S. dollars, and the compound annual growth rate for the five-year period was 7.6\%. By 2016, revenue will be as high as \$51.5 billion (as shown in Figure.~\ref{pic1}). These discoveries throughout the world, and especially in the United States, reflect the rapid adoption of e-learning in the world and emergency alternatives from traditional courses. As a result, e-learning is rapidly becoming mainstream and has become the main method of post-secondary education.
	\footnote{from"A Survey of Artificial Intelligence Techniques Employed for   Adaptive Educational  Systems Within E-Learning Platform	"} 

\begin{figure}[htp]
	\centering
	\includegraphics[width=9cm]{figure1.jpg}
	\caption{\bfseries{ Worldwide Self-paced Five-year Growth Rates by Region}}\label{pic1}
\end{figure}


%\footnote{from"Network newspaper"} 
	\begin{thebibliography}{99}
%	\bibitem{pa}   Newell A, Simon H A. Human Problem Solving. Englewood Cliffs, NJ: Prentice-Hall, 1972 
%	\bibitem{pa}  Minsky M L. The Society of Mind. New York: Simon and Schuster, 1986 
	\bibitem{pa}  T.Kidd,OnlineEducationandAdultLearning.New York: Hershey, 2010.
	\bibitem{pa}    Khalid Almohammadi \emph{etal.} A Survey of Artificial Intelligence Techniques Employed for  Adaptive Educational Systems Within E-Learning Platform, [J] .2017,46--64.
\end{thebibliography}
\end{document}