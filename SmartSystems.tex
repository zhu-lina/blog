\documentclass[a4paper,12pt,twocolumn]{article}
\usepackage{graphicx}
%\usepackage{colorlinks=true]{hyperref}
	\usepackage{indentfirst}
	\setlength{\parindent}{3em}
	\setlength{\parskip}{1em}
\pagestyle{plain}
\linespread{1.5}
\begin{document}
	\begin{center}
		
		{\bfseries \LARGE 	Knowledge of Smart Systems} 
	\end{center}
	\begin{center}
		Lina Zhu
	\end{center}
	\begin{center}
		\today
	\end{center}
	\par How is the subject's “smart strategy” generated? This is clearly the fundamental issue that all intelligent research must answer. According to the previous article that understands Fig.~\ref{pic1}, the global abstract model of the intelligent system represented in Fig.~\ref{pic1} needs to be further studied. Expanding, it becomes a still global but yet deeper and more intrinsic intelligent system model that can comprehensively and profoundly reveal how the “object information” of the object stimulates the “intelligent strategy” and “smart behavior” of the subject's mystery. This results in the intelligent system principle model shown in Fig.~\ref{pic2}. It can both characterize the working mechanism of the human intelligence system and the working principle of the artificial intelligence system. The model of Fig.~\ref{pic2} looks complicated and involves perception. Complex functions such as selection, cognition, basic awareness, emotion, intelligence, comprehensive decision-making and execution. Comprehensively demonstrate the model of the intelligent system. The principle model of Fig.~\ref{pic2} shows that the problem of artificial intelligence research is the open and complex information system. The "mechanical restoration" methodology characterized by the "divide and conquer each defeat" is no longer able to meet the needs of artificial intelligence research, but to "convert information and the "information ecology" methodology characterized by intelligent creation was adapted to the nature of artificial intelligence research and opened up broad prospects for the development of artificial intelligence research. This is the basic trend in the development of artificial intelligence science and technology.
\par	Recalling that the development of artificial intelligence science and technology in the world today can be clearly seen: The most pressing needs and the most rare opportunities in the current artificial intelligence field are the breakthroughs and innovations in the basic theory of artificial intelligence, and this is the potential of Chinese civilization. This is the great opportunity that the Chinese nation faces in the fierce competition between high-tech and cutting-edge science and technology in the contemporary world.
\footnote{from"China Science Magazine"} 
	\begin{figure}[htp]
	\centering
	\includegraphics[width=8cm]{figure1.jpg}
	\caption{\bfseries{ The abstract and global model for intelligent system}}\label{pic1}
\end{figure}
\begin{figure}[htp]
	\centering
	\includegraphics[width=9cm]{figure2.jpg}
	\caption{\bfseries{  The model of intelligent system in principle}}\label{pic2}
\end{figure}


%\footnote{from"Network newspaper"} 
	\begin{thebibliography}{99}
	\bibitem{pa}   Newell A, Simon H A. Human Problem Solving. Englewood Cliffs, NJ: Prentice-Hall, 1972 
	\bibitem{pa}  Minsky M L. The Society of Mind. New York: Simon and Schuster, 1986 
	\bibitem{pa}    Minsky M, Papert S. Perceptron. Cambridge, MA: The MIT Press, 1969 
	\bibitem{pa}  Kosko B. Adaptive bidirectional associative memories. Appl Opt, 1987, 26: 4947-–4960  
\end{thebibliography}
\end{document}