\documentclass[a4paper,12pt,twocolumn]{article}
\usepackage{graphicx}
	\usepackage{indentfirst}
	\setlength{\parindent}{3em}
	\setlength{\parskip}{1em}
\pagestyle{plain}
\linespread{1.5}
\begin{document}
	\begin{center}
		
		{\bfseries \LARGE  Network Security} 
		
	\end{center}
	\begin{center}
		Lina Zhu
	\end{center}
	\begin{center}
		\today
	\end{center}
	
	\par With the development of computer network technology, people hope that a good network environment is free from external interference and destruction, and solving the problem of network security and reliability is particularly important. Network security is shown in Fig.1, in the era of rapid development of information technology, computer networks have been widely used, but at the same time, the amount of network information transmission has increased dramatically, causing network security problems.
	\begin{figure}[htp]
		\centering
		\includegraphics[width=5cm]{figure1.jpg}
		\caption{\bfseries{Network Security}}\label{pic1}
	\end{figure}
\par According to the US FBI's survey, the United States has lost more than \$17 billion in economic losses every year because of the network security. 75\% of companies report that financial losses are caused by computer system security issues. More than 50\% of security threats come from the inside, and only 59\% of losses can be estimated quantitatively. In China, the amount of economic losses caused by the security problems of computer systems in the financial sectors such as banks and securities has amounted to several hundred million yuan, and network security threats against other industries have also occurred from time to time.
\par	No matter what kind of operation is harmful to network security, it will bring immeasurable losses to the system. Therefore, computer networks should have security measures to prevent the network security problems, we jointly build a harmonious network environment.
%\footnote{from"the Journal of Personality and Social Psychology"} 
	\begin{thebibliography}{99}
	\bibitem{pa}   Cai Wangdong, computer network technology. Xian University press, 1988.
	
	\bibitem{pa}   Du Feilong, Internet principle and application, people's posts and telecommunications press, 1997.
	\bibitem{pa}   Network newspaper, 1997-2001.
\end{thebibliography}
\end{document}
