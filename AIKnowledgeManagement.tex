\documentclass[a4paper,12pt,twocolumn]{article}
\usepackage{picinpar,graphicx}
	\usepackage{indentfirst}
		\usepackage{cite}
			\usepackage{multirow}
				\usepackage{CJK}
					\usepackage{caption2}
	\setlength{\parindent}{3em}
\setlength{\parskip}{1em}
\pagestyle{plain}
\linespread{1.5}
\twocolumn
\bibliographystyle{unsrt} 
%\usepackage{multicol} 
 \newenvironment{figurehere}{\def\@captype{figure}}{} 
 \usepackage[colorlinks,linkcolor=red,citecolor=green]{hyperref} 

\begin{document}
%\begin{center}
%	\textbf{\bfseries \LARGE Adaptive Education System AI Technology} 
%\end{center}
%\begin{center}
%	Lina Zhu
%\end{center}
%\begin{center}
%	\today
%\end{center}
 \title{\textbf{\bfseries \LARGE Artificial Intelligence in Knowledge Management} }
\author{\textbf{Lina Zhu}}
\date{\textbf{19 May 2018}}
\maketitle
	\par From this paper, it is concluded that organizations that have succeeded in the global information society are organizations that can identify, evaluate, create, and develop their knowledge assets. Information and Communication Technology (ICT) supports knowledge management. Without a degree of automation, contemporary knowledge management (KM) systems will not succeed. The life cycle of data information knowledge management tasks should be as short as possible to add value to the data to meet the information needs of organizations and individuals. Today, ICTs are often accompanied by artificial intelligence (AI) capabilities to take full advantage of the benefits of these technologies. Artificial intelligence methods are widely used in pattern recognition, mathematical logic, search heuristics, and many other fields. Recently, KM has also received increasing attention as a problem area that can be applied as an artificial intelligence method. Advanced artificial intelligence technologies such as neural networks, genetic algorithms, and intelligent agents provide intelligent tools such as semantic text analysis, text mining, user analysis, and pattern matching. In other areas, these functions are required in the KM task. Through an integrated knowledge management system (KMS) solution, artificial intelligence-based technologies provide support for organization-wide knowledge processing. From the perspective of AI, knowledge represents the normalization of research knowledge and its processing within the machine. Automatic reasoning allows the computer system to draw conclusions from knowledge that can be derived from machine-interpreted forms.
\section{Challenges} 
What has changed since the emergence of challenges in the early 21st century? Is the expectation of the AI ​​and KM alliances achieved? As the publications in the KM field suggest, they have, at least to some extent. Some systems integrate multiple AI technologies. Trying to measure the value of knowledge has brought some results. It is not uncommon for systems to improve their results. In the context of the latest knowledge management trends, Section VI A discusses current efforts to represent tacit knowledge. Figure.~\ref{pic1} shows an AI-related technology example for resolving the previously listed KM drawbacks.
\begin{figure}[htp]
	\centering
	\includegraphics[width=9cm]{figure1.jpg}
	\caption{\bfseries{  Challenges for AI in KM have been addressed by several AI-based or AI-supported technologies  }}\label{pic1}
\end{figure}
\section{Solution}
Table.~\ref{table1} provides a brief overview of all the knowledge management requirements collected in this paper and their latest AI- or artificial-intelligence-based solutions, and describes the challenges of setting up AI in knowledge management.

\section{Conclusion}
Obviously, organizational or personal knowledge cannot be managed with a single technology. KM has many tasks and needs a series of techniques to execute them. These technologies are widely equipped with AI facilities to achieve the desired results. This article collects the latest results of artificial intelligence in the handling of knowledge management tasks. It provides in-depth insight into smart tools that can be used to manage knowledge. It is important to understand the available resources for selection so as not to waste energy creating content that has already been introduced.
		\footnote{from"Scientific Journal of Riga Technical University Computer Science	"} 

%\bibliography{1}
 
%\footnote{from"Network newspaper"} 
\begin{thebibliography}{0}
%	\bibitem{pa}   Newell A, Simon H A. Human Problem Solving. Englewood Cliffs, NJ: Prentice-Hall, 1972 
%	\bibitem{pa}  Minsky M L. The Society of Mind. New York: Simon and Schuster, 1986 
	\bibitem{Carvalho.{2011}} Carvalho, R.B., Tavares Ferreira, M.A. Knowledge Management Software. In: Encyclopedia of Knowledge Management.  2nd ed., 2011, pp. 738-749. 
	\bibitem{Nonaka.{1995}}     Nonaka, I., Takeuchi, H. The knowledge creating company. Oxford Press, New York. 1995. 
\bibitem{Hoffman.{2008}}   Hoffman, R.R., et.al. Knowledge Management Revisited. Intelligent Systems. May/June 2008, pp. 84-87.
\bibitem{Baeshen.{2008}}      Baeshen, N.M.S. Knowledge Management and Environmental Decision Support Systems. In: 12th WSEAS International Conference on Computers. 2008, pp. 793-798. 
\end{thebibliography}
\onecolumn
\begin{table}[!hbp]
	\centering 
	\caption{Classification of AI technology}\label{table1}
	\begin{tabular}{|c|c|}
		\hline
		KM need & Latest attempts to solve the need  \\
		\hline
		Integrated framework for AI technologies  & Environmental decision support systems\cite{Baeshen.{2008}} \\
		\hline
		Measurement of KM benefits & Knowledge loss risk assessment\cite{Hoffman.{2008}} \\
		\hline
		Hybrid KMS that improve their performance  & CBR\cite{Carvalho.{2011}} \\
		\hline
		Elicitation of tacit knowledge & Expert systems\cite{Nonaka.{1995}}\\
		\hline
	\end{tabular}
\end{table}

\end{document}