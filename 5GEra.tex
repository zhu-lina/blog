\documentclass[a4paper,12pt,twocolumn]{article}
\usepackage{graphicx}
%\usepackage{colorlinks=true]{hyperref}
	\usepackage{indentfirst}
	\setlength{\parindent}{2.8em}
	\setlength{\parskip}{0.8em}
\pagestyle{plain}
\linespread{1.5}
\begin{document}
	\begin{center}
		
		{\bfseries \LARGE 5G Era} 
	\end{center}
	\begin{center}
		Lina Zhu
	\end{center}
	\begin{center}
		\today
	\end{center}
	\par As shown in Figure.\ref{pic1}, by the 5G era, if it is possible to use a wider frequency band than the previous mobile broadband, it is necessary to open up new frequency bands available in the future. The 5G available frequency bands are as follows.
		\begin{figure}[htp]
		\centering
		\includegraphics[width=5cm]{figure1.jpg}
		\caption{\bfseries{Network Security Information}}\label{pic1}
	\end{figure}
	\par Low frequency band below 1GHz: Due to its long wavelength and long-distance transmission, it can be used in mobile broadband applications requiring wide-area coverage, and using a large number of low-power IoT sensors in underground and indoor environments.
	\par 1G to 6GHZ medium frequency band: It can ensure large bandwidth and can send large amounts of data. It is also suitable for mission-critical delivery of smart cities, factories, and medical devices.
\par 	High frequency band above 24 GHz (millimeter wave): It can ensure large bandwidth and can send data up to several Gbps. Although the distance of arrival is short, it is highly linear and suitable for connection with specific devices. Therefore, it is suitable for mobile broadband in densely populated places.
\par	On the other hand, the use of frequency bands classified by subordinates is also discussed.Authorized frequency band: Allows mobile communication network operators to use, and can provide licensed frequency bands.Shared Band: This is a frequency band that is guaranteed for other purposes, but can be used when not in use or when it is open.Unlicensed band: A band that can be used without a license. In a limited area, it can be used together with various technologies such as wireless LAN, LTE-related technologies, and Bluetooth.
\footnote{from"Network newspaper"} 


%\footnote{from"Network newspaper"} 
%	\begin{thebibliography}{99}
%	\bibitem{pa}   Hu Daoyuan, Information network system integration technology, the Qinghua University press, 1995.
	
%	\bibitem{pa}   Yang mingfu, Computer network, electronics industry press, 1998.5.
%	\bibitem{pa}   Yuan Baozong, Internet and application,Jilin University press, 2000.
%\end{thebibliography}
\end{document}