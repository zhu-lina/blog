\documentclass[a4paper,12pt,twocolumn]{article}
\usepackage{graphicx}
%\usepackage{colorlinks=true]{hyperref}
	\usepackage{indentfirst}
	\setlength{\parindent}{3em}
	\setlength{\parskip}{1em}
\pagestyle{plain}
\linespread{1.5}
\begin{document}
	\begin{center}
		
		{\bfseries \LARGE 	A Simple Overview of Artificial Intelligence Systems} 
	\end{center}
	\begin{center}
		Lina Zhu
	\end{center}
	\begin{center}
		\today
	\end{center}
	\par Through this paper, we have learned that the intelligence created by natural evolution is called natural intelligence. Therefore, artificial intelligence refers to the intelligence created by humans. The task of artificial intelligence research is to understand the mysteries of natural intelligence, create artificial intelligence machines, and enhance human intelligence. ability. 
	
	\par For the study of natural intelligence and artificial intelligence, there is a profound mystery of thinking and an extremely profound and highly complex study. The methodology popular in the 20th century was the “mechanical reductionism.” Afterwards, people made breakthroughs in structural functions and behaviors. Has achieved very good results. It is understood that the research on artificial neural networks has achieved excellent results in image recognition capabilities such as pattern recognition, fault diagnosis, and deep learning. The physical symbol system has achieved a great deal in enhancing logical thinking capabilities such as the proof of mathematical theorems, information retrieval, and machine games. Exciting achievements; The study of perceptual action systems has made encouraging achievements in enhancing the ability to perform. However, research on artificial intelligence is also facing considerable challenges. We should effectively use intelligent systems in combination with actual situations. Globally and in essence, both human intelligence and artificial intelligence systems can be represented by the macroscopic abstract model of Figure .\ref{pic1}. The "intelligence" of the subject should be a superb ability to exhibit in the process of interacting with the object: First, in the environment The object will apply its “object information” to the subject; then, the intelligent subject will generate a corresponding “smart strategy” according to his own purpose and knowledge and convert it into a “smart behavior” that acts against the object in order to achieve Its own purpose. It is also a simple overview of artificial intelligence systems.
\footnote{from"China Science Magazine"} 

		\begin{figure}[htp]
		\centering
		\includegraphics[width=8cm]{figure1.jpg}
		\caption{\bfseries{ The abstract and global model for intelligent system}}\label{pic1}
	\end{figure}


%\footnote{from"Network newspaper"} 
	\begin{thebibliography}{99}
	\bibitem{pa}  McCulloch W C, Pitts W. A logic calculus of the ideas immanent in nervous activity. Bull Math Biophys, 1943, 5: 115-–133
	\bibitem{pa}  Rosenblatt F. The perceptron: A probabilistic model for information storage and organization in the brain. Psychol Rev, 1958, 65: 386-–408 
	\bibitem{pa}   Hopfield J J. Neural networks and physical systems with emergent collective computational abilities. Proc Natl Acad Sci USA, 1982, 79: 2554-–2558
	\bibitem{pa}  Kohonen T. The self-organizing map. Proc IEEE, 1990, 78: 1464-–1480 
\end{thebibliography}
\end{document}