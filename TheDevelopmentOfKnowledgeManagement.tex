\documentclass[a4paper,12pt]{article}
\usepackage{picinpar,graphicx}
	\usepackage{indentfirst}
		\usepackage{cite}
			\usepackage{multirow}
				\usepackage{CJK}
					\usepackage{caption2}
	\setlength{\parindent}{3em}
\setlength{\parskip}{1em}
\pagestyle{plain}
\linespread{1.5}
%\twocolumn
\bibliographystyle{unsrt} 
%\usepackage{multicol} 
 \newenvironment{figurehere}{\def\@captype{figure}}{} 
 \usepackage[colorlinks,linkcolor=red,citecolor=green]{hyperref} 

\begin{document}
%\begin{center}
%	\textbf{\bfseries \LARGE Adaptive Education System AI Technology} 
%\end{center}
%\begin{center}
%	Lina Zhu
%\end{center}
%\begin{center}
%	\today
%\end{center}
 \title{\textbf{\bfseries \LARGE The Development of Knowledge Management} }
\author{\textbf{Lina Zhu}}
\date{\textbf{21 May 2018}}
\maketitle
	\par The knowledge management is regarded as the four knowledge conversion processes from Nonaka and Takeuchi's framework, namely socialization, externalization, combination and internalization. The knowledge management system based on artificial intelligence mainly supports externalization, composition and internalization. From 2004 to 2007, KM research in the use of artificial intelligence is not very intensive. This work has been active in recent years. The progress made in knowledge management over the past three years shows new directions, and therefore also provides support for smart technology to create space. Smart technologies involve the use of artificial intelligence. The importance of knowledge management is growing, including personal knowledge management and distributed knowledge work (see Figure.~\ref{pic1}). Use different fields, tasks and technical representations.
	\begin{figure}[htp]
		\centering
		\includegraphics[width=9cm]{figure1.jpg}
		\caption{\bfseries{  Topical KM fields (Personal knowledge management and Distributed knowledge work) can be described with specific tasks they involve. These tasks, however, are supported by AI-related technologies.   }}\label{pic1}
	\end{figure}
	\section{Personal knowledge management}
	Although personal knowledge management is not a new topic in itself, it became even more important in the past year. At present, the human ability to obtain information is increasing, but the ability to process information remains unchanged. Personal knowledge management uses computers, communications and networking technologies to help individuals manage information effectively. Managing personal knowledge involves the application of appropriate skills and tools. Diao, Zui, and Liu claim that in order to obtain reliable information quickly and accurately, it is necessary to apply AI to personal knowledge management. Their thesis deals with the following major personal knowledge management issues: information overload, unstructured information, and tacit knowledge. In order to solve these problems, these AI applications were introduced as intelligent search, automatic knowledge classification, and tacit knowledge conversion. Other authors also confirmed the relevance of these issues.
	
	\section{Distributed knowledge work}
	Organizations are extending their traditional coexistence work to virtual work. The situation of dispersal workers increases the demand for communication and collaboration systems and attributes the core functions of ICT and interaction. In such an organization, knowledge will be passed across four boundaries: culture, time, space, and organization. In "virtual work," there is much in common with knowledge mining and search on the Web. The so-called second-generation collaboration tools promote virtual knowledge sharing, which is very important in distributed organizations. Examples of these tools are blogs, wikis and social networking sites. In addition, related personal knowledge management is also related to the organization of distributed work; because the staff of any organization should mainly organize their own knowledge. By implementing new methods of distribution work, technological advances have made new forms of organization possible. The knowledge management tasks highlighted in the next section include the task of enabling personal knowledge management and distributed knowledge work.
\par 	Table.~\ref{table1} provides a brief overview of all the knowledge management requirements collected in this paper and their latest AI or artificial-intelligence-based solutions, describing the most recent tasks. The table supplements references to specific use cases for specific solutions.
\begin{table}[h]%[!hbp]
	%\centering 
	\caption{\bfseries{SUMMARY OF KM NEEDS STATED IN EARLY 2000S AND TOOLS RECENTLY USED FOR SOLVING THEM} }\label{table1}
	
	\begin{tabular}{|c|c|}
		\hline
		KM task & Supporting technologies  \\
		\hline
		Knowledge acquisition in the web   &  Glossaries \\
		\hline
		Hierarchical document classification  &  Multi-label classification, Mulan\cite{Birzniece2011Artificial} \\
		\hline
		Intelligent search   & Intelligent agents \\
		\hline
		Knowledge sharing &Wikis\\
		\hline
	\end{tabular}
\end{table}	


\section{Summary}
	It is clear that the field of knowledge management is increasingly important, including personal knowledge management and distributed knowledge work. These fields are supported by many knowledge management tasks. This article describes most topical knowledge management tasks and their AI or AI-based solutions. The trend of knowledge management is the acquisition of knowledge in the network, the classification of classified documents, and the use of blogs and wikis for intelligent search and knowledge sharing. Paper can also be used as a reference collection for studying specific topics in more detail.
	\footnote{from"Scientific Journal of Riga Technical University Computer Science	"} 
\bibliographystyle{IEEEtran}
\bibliography{ref}  
%\bibliography{1}
 
%\footnote{from"Network newspaper"} 
%\begin{thebibliography}{0}
%	\bibitem{pa}   Newell A, Simon H A. Human Problem Solving. Englewood Cliffs, NJ: Prentice-Hall, 1972 
%	\bibitem{pa}  Minsky M L. The Society of Mind. New York: Simon and Schuster, 1986 
	%\bibitem{Carvalho.{2011}} Carvalho, R.B., Tavares Ferreira, M.A. Knowledge Management Software. In: Encyclopedia of Knowledge Management.  2nd ed., 2011, pp. 738-749. 
%	\bibitem{Nonaka.{1995}}     Nonaka, I., Takeuchi, H. The knowledge creating company. Oxford Press, New York. 1995. 
%\bibitem{Hoffman.{2008}}   Hoffman, R.R., et.al. Knowledge Management Revisited. Intelligent Systems. May/June 2008, pp. 84-87.
%\bibitem{Baeshen.{2008}}      Baeshen, N.M.S. Knowledge Management and Environmental Decision Support Systems. In: 12th WSEAS International Conference on Computers. 2008, pp. 793-798. 
%\end{thebibliography}
%\onecolumn
   
\end{document}