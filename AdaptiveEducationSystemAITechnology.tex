\documentclass[a4paper,12pt,twocolumn]{article}
\usepackage{picinpar,graphicx}
	\usepackage{indentfirst}
		\usepackage{cite}
		%	\usepackage{ctex }
	
			\usepackage{multirow}
				\usepackage{CJK}
					\usepackage{caption2}
	\setlength{\parindent}{3em}
\setlength{\parskip}{1em}
\pagestyle{plain}
\linespread{1.5}
\twocolumn
\begin{document}
%\begin{center}
	
%	\textbf{\bfseries \LARGE Adaptive Education System AI Technology} 
%\end{center}
%\begin{center}
%	Lina Zhu
%\end{center}
%\begin{center}
%	\today
%\end{center}
\title{\textbf{\bfseries \LARGE Adaptive Education System AI Technology} }
\author{\textbf{Lina Zhu}}
\date{\textbf{15 May 2018}}
\maketitle
	\par
	
	The adaptive education system emphasizes the importance of individual differences in modeling the ideal online learning environment. Identifying and catering to the needs and capabilities of individual learners is the key to successful provision of an adaptive e-learning system. Adaptive education systems must have these skills in order to provide their users with appropriate learning methods and content. Therefore, on the basis of analyzing their emotional state, level of knowledge, personality traits and skills, it is very important to create accurate student profiles and models. This data must be used effectively to develop an adaptive learning environment. The data obtained must be effectively used and utilized to develop an adaptive learning environment. As shown in Figure.~\ref{pic1}, these learner models and data can be used in two ways once acquired. The first is the pedagogy that informs experts and designers of adaptive education systems. These reflect the dynamic self-learning ability of teacher student behavior. These learning capabilities will ensure learner and system improvement in lifelong learning mode. Educational data mining and learner analysis are the two main overlapping areas and may be very effective in producing this capability. Artificial intelligence models do not have these capabilities because they can develop and imitate human decision-making processes. Therefore, it is necessary to review and review existing and related topic learning environments, such as large-scale open online courses, educational data mining, learner analysis and related artificial intelligence technologies.
	\begin{figure}[htp]
		\centering
		\includegraphics[width=7cm]{figure1.jpg}
		\caption{\bfseries{	Asynchronous and synchronous learning settings}}\label{pic1}
	\end{figure}
	\par This open learning system of a large number of open online courses can quickly become popular. Any open online course must have two functions: it must be free and open to any user. Recently, some well-known universities and colleges including Stanford University and Harvard University have used open online courses for learning. The number of users that can be registered in any course can range from thousands to millions, that is, they can have a large number of users. The main challenge for open online courses is to collect large data sets from interactions between students and the learning environment and combine them with artificial intelligence techniques to gain insights into adaptive human learning. If we can use these technologies to predict success, then the most important factor is that analysis can be used to power personalized systems that are based on the learner's collection behavior and the preferences of the adaptive behavior of past student data sets. demand. It is understood that the number of students attending large-scale open online courses is high, and the completion rate of these courses is less than 13\%. The analysis of learners driven by adaptive education learning can help solve problems. Learning interaction analysis is not just theory of motivation for learning; it is to help students gain feedback on their performance and learning style. However, despite the ability to learn analytical techniques and tools, there is still a need for artificial interpretation of the data.
	
	\par Educational data mining and learner analysis of various learning management systems can provide e-learning courses. These courses have salient features of a large amount of data available in compiled form. Each step of the process is determined by the student's personal attributes and his performance evaluation. Prepare technology to explore the specialized forms of information received from the education sector and apply the technology to increase understanding of students and their environment for absorbing knowledge. According to the study of learning analysis, learning analysis is defined as the compilation, quantification, analysis, and relationship between information-independent students and their personal characteristics. That is, the learning process can be well understood and improved, along with the environment in which it occurs.
	\footnote{from"A Survey of Artificial Intelligence Techniques Employed for   Adaptive Educational  Systems Within E-Learning Platform	"} 




%\footnote{from"Network newspaper"} 
	\begin{thebibliography}{99}
%	\bibitem{pa}   Newell A, Simon H A. Human Problem Solving. Englewood Cliffs, NJ: Prentice-Hall, 1972 
%	\bibitem{pa}  Minsky M L. The Society of Mind. New York: Simon and Schuster, 1986 
%	\bibitem{pa}  T.Kidd,OnlineEducationandAdultLearning.New York: Hershey, 2010.
	\bibitem{pa}    Khalid Almohammadi \emph{etal.} A Survey of Artificial Intelligence Techniques Employed for  Adaptive Educational Systems Within E-Learning Platform, [J] .2017,46--64.
\end{thebibliography}
\end{document}