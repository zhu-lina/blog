\documentclass[a4paper,12pt,twocolumn]{article}
\usepackage{picinpar,graphicx}
	\usepackage{indentfirst}
		\usepackage{cite}
		\usepackage{cite}
	
			\usepackage{multirow}
				\usepackage{CJK}
					\usepackage{caption2}
	\setlength{\parindent}{3em}
\setlength{\parskip}{1em}
\pagestyle{plain}
\linespread{1.5}
\twocolumn
\bibliographystyle{unsrt} 
\usepackage{indentfirst} 
 \usepackage{amsmath} 
\usepackage{amssymb} 
\usepackage{multicol} 
 \newenvironment{figurehere}{\def\@captype{figure}}{} 
 \usepackage[colorlinks,linkcolor=red,citecolor=green]{hyperref} 
 \DeclareMathOperator*{\argmax}{argmax} 
 \setlength{\parindent}{2em} 

\begin{document}
%\begin{center}
	
%	\textbf{\bfseries \LARGE Adaptive Education System AI Technology} 
%\end{center}
%\begin{center}
%	Lina Zhu
%\end{center}
%\begin{center}
%	\today
%\end{center}
\title{\textbf{\bfseries \LARGE A Survey of Application of Artificial Intelligence Methodology in Adaptive Education System} }
\author{\textbf{Lina Zhu}}
\date{\textbf{15 May 2018}}
\maketitle
	\par 
	According to the above, the purpose of the adaptive education system is to customize the overall learning method to meet the needs of students. Therefore, taking into account their emotional state, level of knowledge, skills, and personality characteristics, it is crucial to accurately create student profiles and models. Therefore, AI methods are considered as evaluative tools, although they have the ability to develop and replicate the decision-making processes that people use. There are various AI technologies in the adaptive education system, such as fuzzy logic and so on (Table.~\ref{table1}). However, most existing adaptive education systems do not learn from the students' behavior. The adaptive education system relies on the ideas of some experts, such as assessment levels.

	
	
	\par Therefore, how do we ensure the assessment of the level of personal knowledge, learning style and other needs of the high accuracy in order to provide the best and correct individual adaptation action? This issue is critical because the uncertainty of student responses is actually assessed through adaptive education methods, as well as the uncertainty associated with the actual understanding and acceptance of the results.
\section{White box method}	
\par	Fuzzy logic was originally proposed by Zadeh in 1965 \cite{L. A. Zadeh.{1965}}and it quickly became a popular and effective technique for user modeling because it can mimic human reasoning. Fuzzy logic can be seen as an extension of traditional set theory, because statements can be part of the truth between absolute truth and absolute falsehood. The fuzzy logic system (FLS) includes four stages (as shown in Figure.~\ref{pic1}): fuzzfier, rule base, inference engine, and \emph{defuzzifier}\cite{ J. Mendel.{1995}}.Rules can be extracted from digital data provided by experts. When building a rule, the FLS can be thought of as a mapping from a simple input to a clear output. Such a mapping can be numerically expressed as y = f(x). Learning and teaching behavior is expressed in a human-readable and linguistically way through fuzzy rules. Their transparency allows them to be quickly evaluated to explain the causes and methods of certain combinations of input-driven specific rules, in which a set of output conclusions have been produced. There is a relationship with the language tags that appear in the system as results and antecedents of rules for grouping input and output values. Based on system-generated data and student interactions, the fuzzy set of FLS can be adjusted. A new category and extension of fuzzy systems can be defined as type 2 FLS, where type 2 fuzzy sets are used to express the uncertainty of numbers and languages. This type2 fuzzy system can be proposed to directly model and reduce its impact Uncertainty. The extended part of type 1 FL is called type 2 FL, and type 2 FL is to be minimized to type 1 FL, and the uncertainty completely disappears. The second type of FLS has more degrees of freedom than the first type of FLS. Type 2 FLS provides a way to deal with the different sources of digital and linguistic uncertainty that exist in the e-learning environment.	
	\begin{figure}[htp]
	\centering
	\includegraphics[width=7cm]{figure1.jpg}
	\caption{\bfseries{ Block diagram of a Fuzzy Logic System }}\label{pic1}
\end{figure}
\section{Decision Trees}	
	\par A decision tree is a tree where each branch node represents a choice between multiple alternatives and each leaf node represents a decision. Decision trees are often used to obtain decision information. The decision tree starts with the root node where the user takes action. From this node, the user recursively partitions each node according to a decision tree learning algorithm. The end result is a decision tree where each branch represents a possible decision plan and its outcome. Importantly, decision tree technology has been proposed to effectively ensure that individual needs are met and learning efficiency is improved, especially in an e-learning environment.
		\footnote{from"A Survey of Artificial Intelligence Techniques Employed for   Adaptive Educational  Systems Within E-Learning Platform	"} 

%\bibliography{1}
 
%\footnote{from"Network newspaper"} 
\begin{thebibliography}{0}
%	\bibitem{pa}   Newell A, Simon H A. Human Problem Solving. Englewood Cliffs, NJ: Prentice-Hall, 1972 
%	\bibitem{pa}  Minsky M L. The Society of Mind. New York: Simon and Schuster, 1986 
	\bibitem{L. A. Zadeh.{1965}}  L. A. Zadeh, Fuzzy sets, Inf. Control, vol. 8, pp. 338–-353, 1965.
	\bibitem{J. Mendel.{1995}}     J. Mendel, Fuzzy logic system for engineering: A tutorial, Proceedings of the IEEE, 1995, vol. 83, no. 3, pp. 345–-374. 

\end{thebibliography}
	\twocolumn
\begin{table}[!hbp]
	\begin{tabular}{|c|c|c|c|c|c}
		\hline
		\hline
		AI-1 & AI-2 & AI-3 & AI-4 & AI-5 & AI-6\\
		\hline
		Fuzzy Logic (FL) & Decision tree & Bayesian networks & Neural Networks & Genetic algorithms & Hidden Markov Models\\
		\hline
	\end{tabular}
	\caption{Classification of AI technology}\label{table1}
\end{table} 
\end{document}