\documentclass[a4paper,12pt,twocolumn]{article}
\usepackage{graphicx}
%\usepackage{colorlinks=true]{hyperref}
	\usepackage{indentfirst}
	\setlength{\parindent}{3em}
	\setlength{\parskip}{1em}
\pagestyle{plain}
\linespread{1.5}
\begin{document}
	\begin{center}
		
		{\bfseries \LARGE  Network Security Features} 
		
	\end{center}
	\begin{center}
		Lina Zhu
	\end{center}
	\begin{center}
		\today
	\end{center}
	
	\par The last time I mentioned the issue of network security, it is important to build a good network environment. As shown in Figure.\ref{pic1}, we learned in this section that the functions of network security include eight parts: access control, security vulnerabilities, attack monitoring, encrypted communication, authentication, backup and recovery, multi-layer defense, and establishment of a security monitoring center. Access control is the access control system established through a specific network segment and services, which prevents most attacks from reaching the target. Checking security vulnerabilities is a periodic inspection of security vulnerabilities. Attack monitoring is an attack monitoring system established for services on specific network segments. , Most attacks can be monitored in real time. Encrypted communication is used to actively encrypt communication. Attackers cannot understand the modification of sensitive information. Authentication is to prevent attackers from impersonating legitimate users. Backup and recovery can recover data as soon as the attack results in loss. And system services; multi-layered defense is an attacker to delay or block its arrival at the attack target after breaking through the first line of defense; setting up a security monitoring center is to provide security system management, monitoring, protection and emergency services for information systems.
	\footnote{from"Chinese Journal of Computers"} 
	\begin{figure}[htp]
		\centering
		\includegraphics[width=5cm]{figure1.jpg}
		\caption{\bfseries{Network Security Information}}\label{pic1}
	\end{figure}
\par fIn short, we should rationally use the functions of network security, and it is also a comprehensive understanding of computer network security, which provides great help in understanding and understanding network knowledge in the future.
\footnote{from"Network newspaper"} 
	\begin{thebibliography}{99}
	\bibitem{pa}   Hu Daoyuan, Information network system integration technology, the Qinghua University press, 1995.
	
	\bibitem{pa}   Yang mingfu, Computer network, electronics industry press, 1998.5.
	\bibitem{pa}   Yuan Baozong, Internet and application,Jilin University press, 2000.
\end{thebibliography}
\end{document}